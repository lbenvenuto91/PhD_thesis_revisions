\chapter[Conclusions and perspectives]{\centering \begin{normalsize} \begin{Huge}
		Conclusions and perspectives
		\end{Huge} \end{normalsize}}
\label{ch:conclusions}


The work carried out in this thesis focuses on the possibility to increase the robustness of GNSS positioning from Android devices. Starting from the current state of art of smartphone-based GNSS positioning (chapter \ref{ch:SoA}), some preliminary tests were carried out in different conditions with an Android device in order to have and assessment of the positioning quality of such devices (chapter \ref{ch:quality_analisys}). Those tests highlight the poor quality of the positioning from smartphones, which presents very noisy observables, and are likely to suffer from multipath and other interferences.
Starting from this consideration, the main contribution of this thesis is the design and implementation of a prototype system for applying multipath mitigation in RTK positioning for Android devices. The proposed systems (chapter \ref{ch:service_devel}) are based on: 
A smartphone device having a dedicated Android App for collecting and organising raw data;
A base station composed by a low-cost GNSS receiver; A remote server based on a modified version of the open source library RTKLIB for the positioning computation.
%\end{itemize}

The developed system works in real time and not only enables RTK positioning for smartphones, but also applies a multipath mitigation strategy. This strategy is based on the MDP algorithm, conceived and patented by Gter, which was implemented in the RTKLIB version adopted in this work. Furthermore in the ambit of this work the MDP algorithm was improved and adapted for working with GNSS observables from Android devices (chapter \ref{ch:service_devel}). The performance of the MDP algorithm were tested in a static and in a kinematic case study (chapter \ref{ch:mdp_results}). The MDP algorithm showed promising results in multipath detection and mitigation, especially in static condition. From the static dataset, it comes clear in fact that the solution accuracy is improved for all the tested configurations. Furthermore, the higher improvements were observed during the multipath induced interval, confirming the algorithm capability to recognise and mitigate this effect. The algorithm not only increases the accuracy, but also eliminates some false fixed positions, increasing then the solution robustness.
Nevertheless, the MDP algorithm detection capabilities need to be studied in deeper detail, especially for Android GNSS receivers that present very noisy observables. Regarding this topic, in this work both a static and adaptive MDP threshold were tested. Concerning the static MDP threshold, it was observed that the performance of the algorithm improves as the threshold value decreases. The optimal threshold value found, for both the datasets is 2.5 m. Being the multipath effect identified by outliers in the MDP trend (figure \ref{FIG:test4mdp_mdpxiaomiublox}a and \ref{FIG:test3mdp_mdpxiaomistonex}a) and also considering the high noise of smartphone's GNSS observables, which makes the MDP variable higher in modulus, the observables having MDP values lower than 2.5 m can not be considered affected by multipath. The adaptive MDP threshold was also tested showing interesting results. In this case each observable has its own threshold value, and multipath identification seems to be more effective. Lastly the effect of SNR were tested for the detection part of the MDP algorithm combining SNR and MDP thresholds by means of two different strategies, i.e. criterion 2 and 3. In criterion 2, the usage of SNR brings benefits to the detection capabilities of the algorithm. The solution accuracy and robustness are increased, especially when high value of SNR threshold (i.e. 35 DB-Hz) are considered. In criterion 3, usage of SNR brings benefits to the detection capabilities of the algorithm only when low SNR threshold are considered (i.e. $<$ 30 DB-Hz). If high SNR threshold are considered in this case, many observables are considered affected by multipath even if they are not. This compromise not only the detection, but also the mitigation performance of the MDP algorithm. For this reason, dealing with very noisy GNSS observables, such as the smartphone's ones, the criterion 3 shouldn't be considered.

The obtained results are  very encouraging and promising. In particular, they confirm that the proposed solution is capable of increasing both accuracy and robustness in  RTK positioning from Android devices. Nevertheless, the validation process of this solution shall continue applying the MDP algorithm in several other tests conducted considering a wider range of boundary conditions.  

Among the future developments of this work it's worth mentioning a deeper investigation of adaptive version of  the MDP algorithm and, in particular, to deal with epochs with several missing observations for some specific satellite.

Another future development of this work is related to  GNSS and INS (Inertial Navigation System) integration using data from the inertial sensor embedded in the smartphone. Integrating the positioning derived from the application of the MDP algorithm with inertial data through a loosely coupled approach should further increase the robustness of the resulting solution. 

The architecture developed in this work lays the foundations for future studies in this field. The App developed in this thesis, in fact, is prepared not only for GNSS data acquisition but also for the acquisition of data coming from other sensors embedded in the smartphone, such as accelerometer, gyroscope and magnetometer. Only minor modifications to the app are needed in order to extend the message format with the additional sensor data. For the processing task, the library Loose-GNSS-IMU\footnote{\url{https://github.com/aaronboda24/Loose-GNSS-IMU}} could be used to integrate RTKLIB.
This library in fact, works well with data coming from RTKLIB, and only requires some modifications to work in real-time.

Finally, there exist different projects aimed at porting RTKLIB on Android devices (e.g. RTKGPS+\footnote{\url{https://github.com/eltorio/RtkGps}} and others). However most of them are not completely tested or cover only a subset of the RTKLIB library.
For this reason, porting our extension of RTKLIB to Android requires additional engineering work that can be seen as a possible future direction for our work.

