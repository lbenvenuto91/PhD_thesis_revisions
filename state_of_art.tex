%\chapter[State of art]{\begin {Huge}\textit{\bf{State of art}} \end{Huge}}

\chapter[Code]{\centering \begin{normalsize} \begin{Huge}
			Code
		\end{Huge} \end{normalsize}}
\label{ch:code}

\section{RTKLIB CODE}
The following sections include the functions and scripts.

\begin{lstlisting}[language=c]
	
	#include "../rtklib.h"
	#include<stdbool.h>
	
	#define AGSYNC1    78        /* gter message sync code 1 "R"  N 78*/
	#define AGSYNC2    66        /* gter message sync code 2 "a"  B 66*/
	
	#define SPEED_OF_LIGHT 299792458.0  // [m/s]
	#define GPS_WEEKSEC 604800  // Number of seconds in a week
	#define NS_TO_S 1e-9 
	#define BDST_TO_GPST 14 //Leap seconds difference between BDST and GPST
	#define DAYSEC 86400
	#define CURRENT_GPS_LEAP_SECOND 18
	#define GLOT_TO_UTC 10800  // Time difference between GLOT and UTC in seconds
	#define ADR_STATE_VALID 0x00000001
	#define STATE_GAL_E1C_2ND_CODE_LOCK 0x00000800
	#define STATE_GAL_E1B_PAGE_SYNC 0x00001000
	#define STATE_CODE_LOCK 0x00000001
	#define STATE_TOW_DECODED 0x00000008
	#define STATE_GLO_TOD_DECODED 0x00000080
	#define STATE_MSEC_AMBIGUOUS 0x00000010
	#define STATE_GAL_E1BC_CODE_LOCK 0x00000400
	#define STATE_GAL_E1C_2ND_CODE_LOCK 0x00000800
	
	
	#define U1(p) (*((uint8_t *)(p)))
	#define I1(p) (*((int8_t  *)(p)))
	static uint16_t U2(uint8_t* p) { uint16_t u; memcpy(&u, p, 2); return u; }
	static uint32_t U4(uint8_t* p) { uint32_t u; memcpy(&u, p, 4); return u; }
	
	typedef struct {
		
		char typemeas[3];
		long long int utcTimeMillis;
		long long int TimeNanos;
		signed long long int FullBiasNanos;
		double BiasNanos;
		int TimeOffsetNanos;
		int State;
		int Svid;
		int ConstellationType;
		long long int ReceivedSvTimeNanos;
		double AccumulatedDeltaRange;
		double Cn0;
		double CarrierFrequencyHz;
		int ADRState;
		double PseudorangeRateMetersPerSecond;
		
	}androidgnssmeas;
	
	static double asc2dbl(int digits, uint8_t* input)
	{
		double retVal = 0.0;
		for (int i = 0; i < digits; i++)
		{
			double tmp = (input[i] - 48);
			for (int j = 1; j < digits - i; j++)
			tmp *= 10.0;
			
			retVal += tmp;
		}
		return retVal;
	}
	
	static long asc2long(int digits, uint8_t* input)
	{
		long retVal = 0;
		for (int i = 0; i < digits; i++)
		{
			long tmp = (input[i] - 48);
			for (int j = 1; j < digits - i; j++)
			tmp *= 10;
			
			retVal += tmp;
		}
		return retVal;
	}
	
	static double R8u(uint8_t* p)
	{
		double retVal = 0;
		for (int i = 0; i < 8; i++)
		retVal += (p[i] * pow(16, i * 2));
		
		return retVal;
	}
	
	static double R8s(uint8_t* p)
	{
		double fact = 1.0;
		double delta = 0.0;
		if ((p[7] & 128) == 128)
		{
			for (int i = 0; i < 8; i++)
			p[i] = 255 - p[i];
			
			fact = -1.0;
			delta = 1.0;
		}
		
		double retVal = 0;
		for (int i = 0; i < 8; i++)
		retVal += (p[i] * pow(16, i * 2));
		
		return (retVal + delta) * fact;
	}
	
	static int sync_gterAndroid(uint8_t* buff, uint8_t data)
	{
		// code here to sync stream with start of next message
		buff[0] = buff[1]; buff[1] = data;
		return buff[0] == AGSYNC1 && buff[1] == AGSYNC2;
		
		
	}
	
	//LB: series of function needed to compute pseudorange and carrier-phase observation
	
	time_t newDateTime(const int year, const int month, const int date, const int hrs, const int min, const int sec) {
		time_t time;
		struct tm tmStruct = { 0 };
		tmStruct.tm_mday = date;
		tmStruct.tm_mon = month - 1;
		tmStruct.tm_year = year - 1900;
		tmStruct.tm_hour = hrs;
		tmStruct.tm_min = min;
		tmStruct.tm_sec = sec;
		time = mktime(&tmStruct);
		if (time == -1) {
			printf("Data non supportata.");
			exit(1);
		}
		return time;
	}
	
	char* formatDateTime(const time_t mytime) {
		/*Used to printf date object*/
		char* dateTimeStr = malloc(20 * sizeof(char));
		struct tm date = *localtime(&mytime);
		sprintf(dateTimeStr, "%.2d/%.2d/%4d %.2d:%.2d:%.2d", date.tm_mday, date.tm_mon + 1, date.tm_year + 1900, date.tm_hour, date.tm_min, date.tm_sec);
		return dateTimeStr;
	}
	
	double check_week_crossover(double tRxSeconds, double tTxSeconds)
	{
		/*
		Checks time propagation time for week crossover
		: param tRxSeconds : received time in seconds of week
		: param tTxSeconds : transmitted time in seconds of week
		: return : corrected propagation time
		*/
		int del_sec = 0;
		double tau = 0, rho_sec = 0;
		tau = tRxSeconds - tTxSeconds;
		if (tau > GPS_WEEKSEC / 2)
		{
			del_sec = round(tau / GPS_WEEKSEC) * GPS_WEEKSEC;
			rho_sec = tau - del_sec;
			if (rho_sec > 10) {
				tau = 0.0;
			}
			else {
				tau = rho_sec;
			}
		}
		return tau;
	}
	
	double glot_to_gpst(time_t gpst_current_epoch, double tod_seconds)
	{
		/*
		Converts GLOT to GPST
		:param gpst_current_epoch: Current epoch of the measurement in GPST
		:param tod_seconds: Time of days as number of seconds
		:return: Time of week in seconds
		*/
		
		// Get the GLONASS epoch given the current GPS time
		struct tm tmStruct, tmStructtod;
		double tod_sec_frac, tod_sec, tow_sec;
		time_t glo_epoch, glo_td, glo_tod;
		int day_of_week_sec;
		
		tod_sec_frac = modf(tod_seconds, &tod_sec);
		
		tmStruct = *localtime(&gpst_current_epoch);
		tmStruct.tm_hour += 3;
		tmStruct.tm_sec -= CURRENT_GPS_LEAP_SECOND;
		glo_epoch = mktime(&tmStruct);
		
		tmStruct = *localtime(&glo_epoch);
		tmStruct.tm_hour = 0;
		tmStruct.tm_min = 0;
		tmStruct.tm_sec = 0;
		tmStruct.tm_sec += tod_sec;
		glo_tod = mktime(&tmStruct); //LB: maybe this passage doent have sense
		
		//The day of week in seconds needs to reflect the time passed before the current day starts
		day_of_week_sec = tmStruct.tm_wday * DAYSEC;
		
		tow_sec = day_of_week_sec + tod_seconds - GLOT_TO_UTC + CURRENT_GPS_LEAP_SECOND;
		
		return tow_sec;
		
	}
	
	const char* get_constellation(androidgnssmeas gnssdata) {
		char constellation[10] = "A";
		
		if (gnssdata.ConstellationType == 1) {
			strcpy(constellation, "G");
			return constellation;
		}
		else if (gnssdata.ConstellationType == 2) {
			strcpy(constellation, "S");
			return constellation;
		}
		else if (gnssdata.ConstellationType == 3) {
			strcpy(constellation, "R");
			return constellation;
		}
		else if (gnssdata.ConstellationType == 4) {
			strcpy(constellation, "J");
			return constellation;
		}
		else if (gnssdata.ConstellationType == 5) {
			strcpy(constellation, "C");
			return constellation;
		}
		else if (gnssdata.ConstellationType == 6) {
			strcpy(constellation, "E");
			return constellation;
		}
		else if (gnssdata.ConstellationType == 7) {
			strcpy(constellation, "I");
			return constellation;
		}
		
	}
	
	const char* getSatID(androidgnssmeas gnssdata) {
		
		char satID[100] = "A";
		
		char number[100] = { 0 };
		_itoa(gnssdata.Svid, number, 10);
		
		strcpy(satID, get_constellation(gnssdata));
		
		strcat(satID, number);
		return satID;
	}
	
	int get_rnx_band_from_freq(double frequency)
	{
		double ifreq = frequency / (10.23E6);
		int iifreq = ifreq;
		if (ifreq >= 154) {
			return 1;
		}
		else if (iifreq == 115) {
			return 5;
		}
		else if (iifreq == 153) {
			return 2;
		}
		else {
			printf("Invalid frequency detected\n");
			return -1;
		}
		
	}
	
	const char* get_rnx_attr(int band, char constellation, int state)
	{
		char attr[10] = "1C";
		
		//Make distinction between GAL E1Cand E1B code
		if (band == 1 && constellation == 'E') {
			if ((state & STATE_GAL_E1C_2ND_CODE_LOCK == 0) && (state & STATE_GAL_E1B_PAGE_SYNC != 0))
			{
				strcpy(attr, "1B");
			}
		}
		else if (band == 5) {
			strcpy(attr, "5Q");
		}
		else if (band == 2 && constellation == 'C') {
			strcpy(attr, "2I");
		}
		return attr;
	}
	
	const char* get_obs_code(androidgnssmeas gnssdata)
	{
		int band, freq;
		char constellation[10] = "A";
		char attr[10] = "A";
		char obscode[10] = "C";
		strcpy(constellation, get_constellation(gnssdata));
		freq = gnssdata.CarrierFrequencyHz;
		band = get_rnx_band_from_freq(freq);
		strcpy(attr, get_rnx_attr(band, constellation, gnssdata.State));
		
		strcat(obscode, attr);
		
		return obscode;
	}
	
	double get_frequency(androidgnssmeas gnssdata)
	{
		double freq;
		if (gnssdata.CarrierFrequencyHz == 0) //LB: check how to parse null values: not sure this case will ever happen
		{
			freq = 154 * 10.24E6;
		}
		else {
			freq = gnssdata.CarrierFrequencyHz;
		}
		return freq;
	}
	
	void check_trck_state(androidgnssmeas gnssdata)
	{
		double freq = get_frequency(gnssdata);
		int freq_band = get_rnx_band_from_freq(freq);
		
		if (gnssdata.ConstellationType == 1 || gnssdata.ConstellationType == 2 || gnssdata.ConstellationType == 4 || gnssdata.ConstellationType == 5)
		{
			if ((gnssdata.State & STATE_CODE_LOCK) == 0)
			printf("State %i, hase STATE CODE LOCK not valid\n", gnssdata.State);
			else if ((gnssdata.State & STATE_TOW_DECODED) == 0)
			printf("State %i, has  STATE TOW DECODED not valid\n", gnssdata.State);
			else if ((gnssdata.State & STATE_MSEC_AMBIGUOUS) != 0)
			printf("State %i, has  STATE_MSEC_AMBIGUOUS not valid\n", gnssdata.State);
		}
		else if (gnssdata.ConstellationType == 3)
		{
			if ((gnssdata.State & STATE_CODE_LOCK) == 0)
			printf("State %i, has STATE CODE LOCK not valid\n", gnssdata.State);
			else if ((gnssdata.State & STATE_GLO_TOD_DECODED) == 0)
			printf("State %i, has STATE_GLO_TOD_DECODED not valid\n", gnssdata.State);
			else if ((gnssdata.State & STATE_MSEC_AMBIGUOUS) != 0)
			printf("State %i, has  STATE_MSEC_AMBIGUOUS not valid\n", gnssdata.State);
		}
		else if (gnssdata.ConstellationType == 6)
		{
			if (freq_band == 1)
			{
				if ((gnssdata.State & STATE_GAL_E1BC_CODE_LOCK) == 0)
				printf("State %i, has STATE GAL E1BC CODE LOCK not valid\n", gnssdata.State);
				else if ((gnssdata.State & STATE_GAL_E1C_2ND_CODE_LOCK) == 0) //State value indicates presence of E1B code
				{
					if ((gnssdata.State & STATE_TOW_DECODED) == 0)
					printf("State %i, has  STATE TOW DECODED not valid\n", gnssdata.State);
					else if ((gnssdata.State & STATE_MSEC_AMBIGUOUS) != 0)
					printf("State %i, has  STATE_MSEC_AMBIGUOUS not valid\n", gnssdata.State);
				}
				else //State value indicates presence of E1C code
				{
					if ((gnssdata.State & STATE_GAL_E1C_2ND_CODE_LOCK) == 0)
					printf("State %i, has STATE_GAL_E1C_2ND_CODE_LOCK not valid\n", gnssdata.State);
					else if ((gnssdata.State & STATE_MSEC_AMBIGUOUS) != 0)
					printf("State %i, has  STATE_MSEC_AMBIGUOUS not valid\n", gnssdata.State);
				}
			}
			else if (freq_band == 5)
			{
				if ((gnssdata.State & STATE_CODE_LOCK) == 0)
				printf("State %i, has STATE CODE LOCK not valid\n", gnssdata.State);
				else if ((gnssdata.State & STATE_GLO_TOD_DECODED) == 0)
				printf("State %i, has STATE_GLO_TOD_DECODED not valid\n", gnssdata.State);
				else if ((gnssdata.State & STATE_MSEC_AMBIGUOUS) != 0)
				printf("State %i, has  STATE_MSEC_AMBIGUOUS not valid\n", gnssdata.State);
			}
			
		}
		else
		printf("Constellation Type %i, is invalid or not implemented\n", gnssdata.ConstellationType);
		
	}
	
	double computePseudorange(androidgnssmeas gnssdata, float psdrgBias)
	{
		int gpsweek = 0;
		double psrange, local_est_GPS_time = 0, gpssow = 0, T_Rx_seconds = 0, T_Tx_seconds = 0, tau = 0, Tod_secs = 0;
		struct tm tmStruct;
		time_t gpst_epoch, gpstime;
		gpstime = newDateTime(1980, 01, 06, 00, 00, 00);
		
		//compute Receiver time
		gpsweek = floor(-gnssdata.FullBiasNanos * NS_TO_S / GPS_WEEKSEC);
		local_est_GPS_time = gnssdata.TimeNanos - (gnssdata.FullBiasNanos + gnssdata.BiasNanos);
		
		gpssow = local_est_GPS_time * NS_TO_S - gpsweek * GPS_WEEKSEC;
		
		tmStruct = *localtime(&gpstime);
		tmStruct.tm_mday += gpsweek * 7;
		tmStruct.tm_sec += gpssow;
		gpst_epoch = mktime(&tmStruct); //LB: Nanoseconds not handled
		
		if (gpst_epoch == -1) {
			printf("Data non supportata.");
			exit(1);
		}
		
		T_Rx_seconds = gpssow - gnssdata.TimeOffsetNanos * NS_TO_S;
		
		//compute satellite emission time
		
		//check trck status
		//check_trck_state(gnssdata);
		
		//split cases depending on different constellations
		if (gnssdata.ConstellationType == 2)
		{
			printf("ERROR: Pseudorange computation not supported for SBAS\n");
			return -1;
		}
		else if (gnssdata.ConstellationType == 3)
		{
			//GLONASS
			Tod_secs = gnssdata.ReceivedSvTimeNanos * NS_TO_S;
			T_Tx_seconds = glot_to_gpst(gpst_epoch, Tod_secs);
		}
		else if (gnssdata.ConstellationType == 5)
		{
			//BEIDOU
			T_Tx_seconds = gnssdata.ReceivedSvTimeNanos * NS_TO_S + BDST_TO_GPST;
			
		}
		else if (gnssdata.ConstellationType == 1 || gnssdata.ConstellationType == 6)
		{
			//GPS and GALILEO
			T_Tx_seconds = gnssdata.ReceivedSvTimeNanos * NS_TO_S;
		}
		else
		{
			printf("Case not implemented\n");
			return -1;
		}
		
		tau = check_week_crossover(T_Rx_seconds, T_Tx_seconds);
		psrange = tau * SPEED_OF_LIGHT;
		return psrange;
	}
	
	double computeCarrierPhase(androidgnssmeas gnssdata) {
		double cphase, wavelength;
		
		wavelength = SPEED_OF_LIGHT / get_frequency(gnssdata);
		
		if ((gnssdata.ADRState & ADR_STATE_VALID) == 0) {
			printf("ADR STATE not Valid --> cphase = 0.0 \n");
			cphase = 0.0;
			return cphase;
		}
		
		cphase = gnssdata.AccumulatedDeltaRange / wavelength;
		
		return cphase;
	}
	
	double computeDoppler(androidgnssmeas gnssdata)
	{
		double doppler, wavelength;
		wavelength = SPEED_OF_LIGHT / get_frequency(gnssdata);
		doppler = -gnssdata.PseudorangeRateMetersPerSecond / wavelength;
		return doppler;
	}
	
	char* mystrsep(char** stringp, const char* delim)
	{
		char* start = *stringp;
		char* p;
		
		p = (start != NULL) ? strpbrk(start, delim) : NULL;
		
		if (p == NULL)
		{
			*stringp = NULL;
		}
		else
		{
			*p = '\0';
			*stringp = p + 1;
		}
		
		return start;
	}
	
	
	
	static int decode_gterAndroid(raw_t* raw)
	{
		// code here to decode android message
		
		//parse
		//   int row_count_r = 0;
		//   int col_count_r = 0;
		//   char* col = strtok(raw->buff, ",");
		
		const char delimiters[] = ",";
		char* running;
		char* currstrval; //current field read in the str	
		int ncolmeas = 30; //number of fields for single measurement
		int nmeas;
		
		
		running = raw->buff;
		
		//read the number of Raw meas sent for this epoch
		for (int i = 0; i < 2; i++) {
			currstrval = mystrsep(&running, delimiters);
			if (i == 1)
			nmeas = atoi(currstrval);
		}
		
		//read values in every Raw meas 
		androidgnssmeas* andrawdata = malloc(sizeof(androidgnssmeas)*nmeas);
		for (int n = 0; n < nmeas; n++) {
			
			for (int j = 0; j < ncolmeas; j++) {
				currstrval = mystrsep(&running, delimiters);
				if (j == 0)
				strcpy(andrawdata[n].typemeas, currstrval);
				else if (j == 1)
				andrawdata[n].utcTimeMillis = atoll(currstrval);
				else if (j == 2)
				andrawdata[n].TimeNanos = atoll(currstrval);
				else if (j == 5)
				andrawdata[n].FullBiasNanos = atoll(currstrval);
				else if (j == 6)
				sscanf(currstrval, "%lf", &andrawdata[n].BiasNanos);
				else if (j == 11)
				andrawdata[n].Svid = atoi(currstrval);
				else if (j == 12)
				andrawdata[n].TimeOffsetNanos = atoi(currstrval);
				else if (j == 13)
				andrawdata[n].State = atoi(currstrval);
				else if (j == 14)
				andrawdata[n].ReceivedSvTimeNanos = atoll(currstrval);
				else if (j == 16)
				sscanf(currstrval, "%lf", &andrawdata[n].Cn0);
				else if (j == 17)
				sscanf(currstrval, "%lf", &andrawdata[n].PseudorangeRateMetersPerSecond);
				else if (j == 19)
				andrawdata[n].ADRState = atoi(currstrval);
				else if (j == 20)
				sscanf(currstrval, "%lf", &andrawdata[n].AccumulatedDeltaRange);
				else if (j == 22)
				sscanf(currstrval, "%lf", &andrawdata[n].CarrierFrequencyHz);
				else if (j == 28)
				andrawdata[n].ConstellationType = atoi(currstrval);
			}
		}
		
		
		
		char* sat;
		char* code;
		float psdrgBias = 0;
		double psdrange,cphase,doppler;
		
		//sat = getSatID(andrawdata);
		
		//code = get_obs_code(andrawdata);
		
		//psdrange = computePseudorange(andrawdata, psdrgBias);
		
		//cphase = computeCarrierPhase(andrawdata);
		
		//doppler = computeDoppler(andrawdata);
		
		//printf("GTER DEBUG GNSS ANDROID: %s, %s", sat, code);
		
		
		
		
		
		
		
		
		
	}
	
	extern int input_gterAndroid(raw_t* raw, uint8_t data)
	{
		trace(5, "input_gterAndroid: data=%02x\n", data);
		
		/* synchronize frame */
		if (raw->nbyte == 0) {
			if (!sync_gterAndroid(raw->buff, data)) return 0;
			raw->nbyte = 2;
			return 0;
		}
		raw->buff[raw->nbyte++] = data;
		
		//TODO:check on possible corrupted message?
		//if (raw->nbyte == 4) {
			//    if ((raw->len = U2(&raw->buff[2])) > MAXRAWLEN) { // warning: modify this
				//        trace(2, "ay length error: len=%d\n", raw->len);
				//        raw->nbyte = 0;
				//        return -1;
				//    }
			//}
		//if (raw->nbyte < 4 || raw->nbyte < raw->len) return 0;
		//raw->nbyte = 0;
		
		if (raw->buff[raw->nbyte - 1] != '\n')
		return 0;
		else if (raw->nbyte == 570 & raw->buff[568] == 'b') {
			
			raw->nbyte = 0;
			return -1;
		}
		raw->nbyte = 0;
		/* decode ublox raw message */
		return decode_gterAndroid(raw);
	}
	
\end{lstlisting}